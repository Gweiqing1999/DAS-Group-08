% Options for packages loaded elsewhere
\PassOptionsToPackage{unicode}{hyperref}
\PassOptionsToPackage{hyphens}{url}
%
\documentclass[
]{article}
\usepackage{amsmath,amssymb}
\usepackage{lmodern}
\usepackage{iftex}
\ifPDFTeX
  \usepackage[T1]{fontenc}
  \usepackage[utf8]{inputenc}
  \usepackage{textcomp} % provide euro and other symbols
\else % if luatex or xetex
  \usepackage{unicode-math}
  \defaultfontfeatures{Scale=MatchLowercase}
  \defaultfontfeatures[\rmfamily]{Ligatures=TeX,Scale=1}
\fi
% Use upquote if available, for straight quotes in verbatim environments
\IfFileExists{upquote.sty}{\usepackage{upquote}}{}
\IfFileExists{microtype.sty}{% use microtype if available
  \usepackage[]{microtype}
  \UseMicrotypeSet[protrusion]{basicmath} % disable protrusion for tt fonts
}{}
\makeatletter
\@ifundefined{KOMAClassName}{% if non-KOMA class
  \IfFileExists{parskip.sty}{%
    \usepackage{parskip}
  }{% else
    \setlength{\parindent}{0pt}
    \setlength{\parskip}{6pt plus 2pt minus 1pt}}
}{% if KOMA class
  \KOMAoptions{parskip=half}}
\makeatother
\usepackage{xcolor}
\usepackage[margin=1in]{geometry}
\usepackage{graphicx}
\makeatletter
\def\maxwidth{\ifdim\Gin@nat@width>\linewidth\linewidth\else\Gin@nat@width\fi}
\def\maxheight{\ifdim\Gin@nat@height>\textheight\textheight\else\Gin@nat@height\fi}
\makeatother
% Scale images if necessary, so that they will not overflow the page
% margins by default, and it is still possible to overwrite the defaults
% using explicit options in \includegraphics[width, height, ...]{}
\setkeys{Gin}{width=\maxwidth,height=\maxheight,keepaspectratio}
% Set default figure placement to htbp
\makeatletter
\def\fps@figure{htbp}
\makeatother
\setlength{\emergencystretch}{3em} % prevent overfull lines
\providecommand{\tightlist}{%
  \setlength{\itemsep}{0pt}\setlength{\parskip}{0pt}}
\setcounter{secnumdepth}{5}
\usepackage{booktabs}
\usepackage{longtable}
\usepackage{array}
\usepackage{multirow}
\usepackage{wrapfig}
\usepackage{float}
\usepackage{colortbl}
\usepackage{pdflscape}
\usepackage{tabu}
\usepackage{threeparttable}
\usepackage{threeparttablex}
\usepackage[normalem]{ulem}
\usepackage{makecell}
\usepackage{xcolor}
\ifLuaTeX
  \usepackage{selnolig}  % disable illegal ligatures
\fi
\IfFileExists{bookmark.sty}{\usepackage{bookmark}}{\usepackage{hyperref}}
\IfFileExists{xurl.sty}{\usepackage{xurl}}{} % add URL line breaks if available
\urlstyle{same} % disable monospaced font for URLs
\hypersetup{
  pdftitle={The Impact of film's features on IMDB Ratings: A logistic Regression approach},
  pdfauthor={Yujie Tang, Jialu FU, Weiqing GUO, Bashiru Mukaila, Wanding Wang},
  hidelinks,
  pdfcreator={LaTeX via pandoc}}

\title{The Impact of film's features on IMDB Ratings: A logistic
Regression approach}
\author{Yujie Tang, Jialu FU, Weiqing GUO, Bashiru Mukaila, Wanding
Wang}
\date{2023-03-14}

\begin{document}
\maketitle

\hypertarget{sec:intr}{%
\section{Introduction}\label{sec:intr}}

The entertainment industry happens to be a competitive market and the
film industry is a subset of it. The stakeholders in the film industry
are interested in features that make their films successful. One key
aspect of a film's success is its rating on platforms such as IMDB,
which can influence audience perception and drive revenue.

The aim of this project is to investigate the relationship between the
properties of films and their IMDB ratings. Specifically, we want to use
a logistic regression model to know which properties of a film influence
whether a film is rated by IMDB as greater than 7 or not, using
variables such as year of release, length, budget, number of votes, and
genre.

Section \ref{sec:expl} speaks to the exploratory data analysis of IMDP
ratings and explores the relationship between the rating and the
properties of the films. Section \ref{sec:form} contains the results
from the logistic regression model. While Section \ref{sec:conc} gives
the final remark to the research.

\hypertarget{sec:expl}{%
\section{Exploratory Data Analysis}\label{sec:expl}}

The data contains 2847 observations and 7 variables. The variables year,
length, budget, votes, and rating are numerical variables, and genre is
a categorical variable. Table \ref{tab:head} shows the first 6 rows of
the table. It is also worth of note that the length of some of the films
is not recorded.

\begin{table}[!h]

\caption{\label{tab:expl3}\label{tab:head} Highlight of the IMDB data}
\centering
\resizebox{\linewidth}{!}{
\fontsize{3}{5}\selectfont
\begin{tabular}[t]{rrrrrlr}
\toprule
film\_id & year & length & budget & votes & genre & rating\\
\midrule
5993 & 1943 & 65 & 15.5 & 42 & Action & 7.6\\
37190 & 1961 & 87 & 12.3 & 6 & Drama & 6.0\\
43646 & 1987 & 79 & 16.4 & 161 & Action & 7.5\\
28476 & 1976 & NA & 12.2 & 5 & Documentary & 8.0\\
23975 & 1982 & 88 & 12.5 & 97 & Action & 3.5\\
\addlinespace
50170 & 1936 & NA & 7.0 & 146 & Drama & 4.4\\
\bottomrule
\end{tabular}}
\end{table}

From the summary statistics in table \ref{tab:sums}, we can see that the
year of release of the films is between 1898 to 2005. The minimum budget
for film in the data set is 2.5 million while the maximum is 22.3
million.

\begin{table}[!h]

\caption{\label{tab:expl4}\label{tab:sums} Summary statistics of the numerical variables}
\centering
\resizebox{\linewidth}{!}{
\fontsize{7}{9}\selectfont
\begin{tabular}[t]{llllll}
\toprule
  &      year &     length &     budget &     votes &     rating\\
\midrule
 & Min.   :1898 & Min.   :  1.00 & Min.   : 2.50 & Min.   :     5 & Min.   :0.800\\
 & 1st Qu.:1957 & 1st Qu.: 73.00 & 1st Qu.: 9.90 & 1st Qu.:    11 & 1st Qu.:3.700\\
 & Median :1982 & Median : 90.00 & Median :11.90 & Median :    29 & Median :4.600\\
 & Mean   :1976 & Mean   : 82.22 & Mean   :11.85 & Mean   :   657 & Mean   :5.342\\
 & 3rd Qu.:1997 & 3rd Qu.:101.00 & 3rd Qu.:13.70 & 3rd Qu.:   114 & 3rd Qu.:7.700\\
\addlinespace
 & Max.   :2005 & Max.   :480.00 & Max.   :22.30 & Max.   :149494 & Max.   :9.200\\
 & NA & NA's   :131 & NA & NA & NA\\
\bottomrule
\end{tabular}}
\end{table}

From the scatterplot in figure \ref{fig:scatp}, we can see that there is
a negative correlation between the length of the film and its rating.
This implies that films with longer length tend to be rated low. There
is a positive correlation between the budget of film production and its
rating, however, the correlation is weak.

\begin{figure}[H]

{\centering \includegraphics[width=0.8\linewidth]{Group_08_Analysis_files/figure-latex/scatp-1} 

}

\caption{\label{fig:scatp} Scatterplot matrix of the numerical variables.}\label{fig:scatp}
\end{figure}

From figure \ref{fig:boxp1}, we can see that some genres tend to have
higher ratings than others. For example, comedy tends to have higher
ratings than romance films. While the majority of the short films are
rated higher than 7.

\begin{figure}[H]

{\centering \includegraphics[width=0.8\linewidth]{Group_08_Analysis_files/figure-latex/boxp-1} 

}

\caption{\label{fig:boxp1} Boxplot of rating by genre.}\label{fig:boxp}
\end{figure}

\hypertarget{sec:form}{%
\section{Formal Data Analysis}\label{sec:form}}

\hypertarget{sec:conc}{%
\section{Conclusion}\label{sec:conc}}

\end{document}
